\section {Background}

\subsection {Bonding Coordination}

A hydrated \suldiox~in an aqueous environment forms hydrogen bonds through the oxygens to nearby water hydrogens, or interacts via the sulfur atom with water-oxygens. To further our analysis of the way in which \suldiox~coordinates its bonding to surface waters, those that lie in the topmost region of a gas/water interface, we adopt a naming scheme to denote the way in which the \suldiox~is hydrated by the surrounding waters. This naming scheme mimics a notational system developed during a previous study on water coordination by Buch et al,\cite{Buch 2005} and was subsequently used in other computational work.\cite{Walker2006b} In this naming system, a letter is used to designate the atom on a water molecule through which a hydrogen bond is formed to neighboring waters. Thus a bonding coordination of ``OOH'' designates two proton-acceptor bonding interactions through the water-oxygen, and a single proton-donor bonding interaction through a hydrogen. More recently, Baer et al. devised a nomenclature that explicitly enumerates the bonding to \suldiox~via the sulfur or oxygen atoms.\cite{Baer2010} 

In this work we adopt Buch et al's nomenclature to \suldiox~in order to quantify hydrogen bonding through the acceptor \suldiox-oxygens, and the weaker bonding interactions from the \suldiox-sulfur to water-oxygens. Thus, an ``SOO'' coordinated \suldiox~molecule forms a single interaction through the sulfur atom to a neighboring water-oxygen, and two hydrogen bonds through either a single \suldiox-oxygen, or distributed with one hydrogen bond on each of the \suldiox-oxygens. Analysis of the distribution of \suldiox~coordinations will give insight to how \suldiox~binds to the water surface.

In our analysis we defined bonds using the distance criteria of Baer et al. in determining the \suldiox~bonding coordinations.\cite{Baer2010} The bond-length definition is based on a set of distance criteria where a bonding interaction between a \wat-oxygen and \suldiox-sulfur is formed at a distance less than 3.5 \angs, and an \suldiox-oxygen hydrogen bond to a \wat-hydrogen is formed at a distance less than 2.2 \angs.

\subsection {Cyclic Bonding Structures}

Hydrated \suldiox~clusters have been studied extensively with many recent experiments and computations forming a clearer picture of \suldiox~bulk and surface behaviors.\cite{Baer2010, Tarbuck2005, Tarbuck2006, Ota2011, Bishenden1998, Hirabayashi2006, Steudel2009, Yang2002, Hayashi1985, Moin2011, Eckl2008} At a water surface, it is now known that \suldiox~forms a complex with water during adsorption, and then subsequently absorbs into the interfacial region by reaction to form ionic sulfur species.\cite{Tarbuck2005, Tarbuck2006, Ota2011} The first attempt to elucidate the structure of surface hydrated \suldiox~by Baer et al. showed that the bonding coordination distributions of \suldiox~are altered relative to the bulk region, and that two coordinations dominate the distribution of bonding types: the ``SO'' and the ``SOO''. In the same work they then focused on the most dominant coordination to determine the most likely cluster geometry using two and three-water clusters bound to the \suldiox.

The cluster geometries found in this study imply a cyclic bonding structure through the two or three waters involved. ``Cyclic'' here is used to denote a closed loop formed by the intermolecular hydrogen bonds, S-O interactions, and covalent bonds of the molecules involved. Figure \ref{fig:cyclic-example} depicts one such cyclic structure showing the bonds beginning on the sulfur and returning through the \suldiox-oxygen.

\begin{figure}[h!]
	\begin{center}
		\includegraphics[scale=1.0]{images/cycles/double-cycle-type2-small.png}
		\caption{During the course of MD, the covalent and intermolecular bonds between a \suldiox~and the hydrating waters may form into a ring, resulting in a cyclic hydrate structure. Depicted here is one example of such a cyclic structure, formed by the covalent SO and OH bonds of the two molecule types, and the intermolecular hydrogen bonds, and S-O$_{H_2O}$ interactions. The atoms involved are the S (yellow), O (red), and H (white).}
		\label{fig:cyclic-example}
	\end{center}
\end{figure}

\subsection {Graph Theoretical Details}
The optimized geometry of the \suldiox-hydrates suggests cyclic bonding structures, but it remains a different story entirely when a \suldiox~is placed in a dynamic environment such as in the course of molecular dynamics simulations of an aqueous surface. Geometry optimization shows the formation of these cyclic hydrate structures with two or three waters in a static vacuum. Do the cyclic structures also form in the course of a dynamic bonding process on a simulated water surface, where extended hydrating structures influence \suldiox~and water behavior? To study the formation and behavior of cyclic hydrate structures we employ graph theoretical techniques on MD trajectory data. Previous use of graphs in molecular computations were applied to finding stable arrangements of water clusters, ice, hydrogen bonding, extracting topological molecular properties, and cyclic structure studies.\cite{Anick2002, Huber2007, Radhakrishnan1991, Shi2005, Garcia2004, McDonald1998}

Here we briefly introduce graph theoretical concepts. They have been described well by others with varied application to cyclic structures.\cite{Tutte1984, Balakrishnan2000, Harary1973, Huber2007, Garcia2004, Dury2001} A graph consists of nodes, and edges that connect the nodes. A molecule can be represented with atoms as nodes, and edges for each intramolecular covalent bond connecting the atoms. The set of edges is then further expanded to include intermolecular interactions such as hydrogen bonds and other bonding bonding interactions. Edges may be assigned weights (i.e. bond lengths), types, and can be directional i.e. pointing towards a target node from a source node. A molecular system including all atoms, bonds, and interactions is thus fully described by a graph. 

To detect cyclic structures in a graph a depth-first or breadth-first search (DFS and BFS, respectively) may be used.\cite{Knuth1997, Cormen 2001} A BFS is a recursive algorithm of queuing nodes and all neighboring nodes while performing a specified procedure on each visited node. This is easily performed on adjacency list or connectivity matrix data structures, iterating through nodes (i.e. atoms) of interest in the graph as starting points of the search. In BFS terminology, all nodes are colored during graph traversal to distinguish unvisited nodes (white), queued nodes (gray), and visited nodes (black). Using a BFS on a graph, cyclic structures are detected any time a ``gray target'' is encountered when queuing adjacent neighbors of a node. A benefit of BFS on a graph is the ability to determine the smallest cyclic structure containing a given node. In the case of \suldiox~hydrate structures, beginning the BFS with the \suldiox~sulfur as the starting, or root node or the search, will discover cyclic bonding structures in order of size. Here we are only concerned with the smallest cyclic structure involving those waters in the first and second hydration shells around the \suldiox. Furthermore, it is possible to reconstruct a cycle's structure by finding its size (number of contributing atoms), and the number of unique waters in the cycle. This allows us to distinguish between various types of cyclic structures encountered.

Several arrangements of cyclic bonding structures are shown in Figure \ref{fig:cyclic-structures} for a \suldiox~molecule with three waters. Cycles with fewer or greater numbers of waters are also possible and encountered during MD. Cycle types \Rmnum{1}, \Rmnum{2}, \Rmnum{3} in Figure \ref{fig:cyclic-structures} are cyclic structures in which the \suldiox~is a member of the cycle. Types \Rmnum{4} and \Rmnum{5} do not involve the \suldiox~in the bonding cycle, but are commonly encountered as the smallest cycle formed near the \suldiox. Type \Rmnum{3} is of particular interest because the \suldiox~in this cycle has the most frequently occuring bonding coordination (``SO'', as shown later).

\begin{figure}[h!]
	\begin{center}
		\includegraphics[scale=1.0]{images/cycles/cycle-types-small.png}
		\caption{\suldiox~in various cyclic structures encountered during MD simulations with water. The cartoons show the five types of cyclic structures, numbered for reference. The cyclic structures may involve any number of waters, but here each structure is shown with three waters. Type \Rmnum{3} is formed by a \suldiox~with the ``SO'' bonding coordination, which is the most dominant bonding coordination encountered.}
		\label{fig:cyclic-structures}
	\end{center}
\end{figure}

Baer et al. presented a detailed geometric and spectroscopic breakdown of type \Rmnum{3} cycles with two and three waters from their DFT calculations.\cite{Baer2010} Given the information of the number of waters, atoms, and bonds involved in the bonding cycles, we find that of the three-water type \Rmnum{3} cycles, there exist two structural varieties, shown in Figure \ref{fig:type-3-varieties}, that differ in the set of water atoms involved in the cyclic structure. Type \Rmnum{3}-A (shown in Figure \ref{fig:type-3-varieties}A) is arranged with each water contributing an OH bond to the structure of the cycle. Type \Rmnum{3}B involves a single water contributing an OH, whereas the other two waters contribute only the oxygen atom or the entire water molecule to the structure, respectively. This nuance of the type \Rmnum{3} structures involving three waters, and the overall distribution are presented in more detail later. We also show the distribution of cyclic structures encountered during MD simulations to further understand the behaviors of \suldiox-hydrates at the water surface.

\begin{figure}[h!]
	\begin{center}
		\includegraphics[scale=1.0]{images/cycles/triple-cycle-types-small.png}
		\caption{Type \Rmnum{3} cyclic structures (see Figure \ref{fig:cyclic-structures}) are found to occur in two varieties that are distinguished by the water atoms contributing to the cyclic structure. In ``type A'' the cycle is formed by an OH bond contributed by each water, and the SO bond of the \suldiox. ``type B'' involves an oxygen atom from one water, an OH bond from a second water, and all atoms of the third water.}
		\label{fig:type-3-varieties}
	\end{center}
\end{figure}
