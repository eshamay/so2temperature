\section {Conclusions}

The adsorption of small gas molecules to water surfaces has been extensively studied over the past few decades. Much has been learned about the energies of hydrate configurations, and the kinetics of gaseous uptake into aqueous systems. Yet, the specific molecular nature of the adsorption process, including the various geometries, hydrate species, and bonding pathways remains largely unknown. As a gas transitions into the liquid water phase, it passes through a fluid interfacial region that remains poorly understood. Our understanding of the processes and chemistry of the interface is still in its infancy, but we are beginning to gain unique new insights that are key to understanding many environmentally important processes at aqueous surfaces.

Presented herein are the results of ab initio molecular dynamics simulations that focus on how a wandering gaseous \suldiox~molecule first makes contact with a water surface, and subsequently forms extended hydrate structures with interfacial water molecules. The computational studies complement and expand on experimental studies from this laboratory that found surface complexation of \suldiox~at a water surface.\cite{Tarbuck2005,Tarbuck2006,Ota2011} Furthermore, these computations build upon and enrich our understanding of adsorbing \suldiox~behavior from our recently published computational study on interfacial geometries of aqueous surface \suldiox~molecules.\cite{Shamay2011}

Our simulations show that \suldiox~has a preferred means of bonding and interacting with surface water molecules by taking on various bonding coordinations. In this work it was shown that the ``SO'' bonding configuration is the most preferred, with ``S'' and ``SOO'' also contributing greatly to the coordination distribution. Once a \suldiox~has bound to form a surface hydrate it rarely forms multiple bonding interactions through the sulfur atom, and even less frequently takes on a configuration with no sulfur interactions to nearby waters.

This study is one of very few temperature studies looking at the microscopic nature of interfacial gas molecules on water. By changing the temperature, it was found that a hotter water system leads to longer \suldiox~binding to the water surface. The distribution of bonding coordinations was greatly affected by a temperature change, shifting populations of bonding configurations because of the altered \suldiox~and \wat~behavior. At the higher temperature, \suldiox~forms more frequent bonds to interfacial waters through the sulfur and oxygen atoms. Overall, we have determined that the \suldiox~hydrate interactions are transient, binding and unbinding to water molecules rapidly in very dynamic bonding coordinations.

In this work we introduce the use of a graph structure to represent atoms and interconnectedness between molecules, and also to represent transitions between the various bonding coordinations of an adsorbed \suldiox. It was shown that the intermolecular bonds formed through the \suldiox-oxygens are quickly broken and formed, lasting briefly compared to sulfur interactions. From the graph of bonding coordination transitions, we found a likely pathway for \suldiox adsorption starting with an unbound gas-phase \suldiox, ending with a hydrated \suldiox~species bound to surface waters.

The formation of cyclic hydrate structures was probed and it was found that these hydrated bonding ring species form during much of a simulated trajectory. Temperature increases the occurrence of cyclic structures, and also shifts the distribution of the specific types of cycles being formed. Two types of cyclic tri-hydrates were discovered during the course of simulations. The cycle lifetimes were found to be mostly short-lived, with a majority lasting less than 1 ps before breaking and reforming due to the dynamic bonding and motion of the surface waters and \suldiox~molecules. Temperature did not have a very dramatic effect on the cyclic lifespans, but higher temperatures did lead to \suldiox~bonding coordinations that are more likely to form into cyclic hydrates.

These studies build upon our computational and experimental research in this area, seeking to understand how gases adsorb and transit across an aqueous/air interface. Such knowledge is invaluable for understanding land water and environmental aerosol systems where gaseous uptake behavior at a water surface surprises us and often defies our physical intuitions.\cite{Jayne1990,Yang2002,Worsnop1989,Boniface2000}
