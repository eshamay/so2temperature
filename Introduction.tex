\section {Introduction}

One of chemistry's many remaining ``black box'' events is the adsorption of a gas into a water surface. Although gas uptake into aqueous systems occurs often environmentally and industrially, we still know very little about the process and the details of the adsorption reactions. How does gas bind to a water surface, and what steps are involved in adsorption? How does a gas molecule near a water surface affect the water to which it will bind? What is the structure of hydrating waters in the surface region, and how does that hydrated solute molecule behave differently than as a gas? Experiments to address these questions provide valuable information, but can never fully describe these microscopic events and behaviors. The information can be determined computationally, and when coupled to the previous experimental work can provide a much more complete picture of gaseous adsorption to aqueous surfaces.

As a model gas, \suldiox~has been used extensively in research because of its importance in commercial and environmental systems. The molecular structure, high solubility in water, and its abundance make it a pivotal compound in numerous aqueous atmospheric reactions. A complete picture of the \suldiox~adsorption process will aide in understanding gaseous adsorption on the many aqueous surfaces in the environment, as well as in understanding the fundamental nature of gases in water interfacial regions. 

In this study we use ab initio quantum molecular dynamics (MD) techniques to model and simulate the hydrating structure that forms around a surface-bound \suldiox~on water. We simulate a dynamic water surface with an extended hydrogen-bonding network that captures the fluidity of the \suldiox~hydrate structures, and the behaviors on the water surface. Unlike small cluster studies that use DFT calculations on geometry optimized structures to find likely geometries in a static vacuum setting, The quantum MD technique allows us to simulate the process much more realistically, and with greater accuracy than our previous classical MD work could allow.\cite{Shamay2011} This computational study greatly enhances the picture developed in our previous work on \suldiox. In our previous computational study we used classical MD to determine net orientational behavior of \suldiox~binding to a water surface, and of the orientation of the waters in response to the presence of an adsorbing gas. Understanding the net behavior of the molecules in the aqueous interfacial region throughout adsorption was the first step to understanding the specific details of gas binding and surface behavior. 

Quantum MD techniques accurately reproduce the hydration geometry around the bound \suldiox~molecules, and allow us to look at the specific bonding interactions that form within the surface hydrates, and in the extended bonding further into the water. The surfaces in this work were simulated at two temperatures: room temperature at 298K, and the more atmospherically relevant cold 270K. This set of temperatures complements our most recent experimental studies that showed the binding of gaseous \suldiox~to a water surface is greatly enhanced at cold temperatures.\cite{Ota2011} Other experiments by our group developed the picture of \suldiox~adsorption, and showed that \suldiox~surface hydrate complexes form when a water surface is exposed to \suldiox~gas.\cite{Tarbuck2005,Tarbuck2006} Conclusions from the experiments regarding the specific nature of those complexes could only be inferred. 

We believe this to be the first temperature study using quantum MD to look at the binding of a small gas molecule to a water surface. We show how temperature affects the bonding behavior of the surface adsorbed \suldiox~to neighboring waters. A proposed set of steps in a binding mechanism for \suldiox~adsorbing to a water surface is presented. We also look at \suldiox~behavior when already bound by the surface waters. Lastly, our analysis shows some interesting extended bonding behavior of \suldiox~hydrates, and that the hydrating waters and the \suldiox~often form cyclic ring structures through intermolecular bonds.
