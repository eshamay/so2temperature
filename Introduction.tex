\section {Introduction}

The molecular nature of the adsorption of gas molecules onto a water surface is one of the remaining largely uncharted territories of surface chemistry. Although gas uptake into aqueous systems occurs often environmentally and industrially, we still know very little about the process and the details of the adsorption reactions, and certainly less than what we know about gaseous adsorption on a solid surface. How does a gas initially bind to a water surface, and what steps are involved in the subsequent adsorption? How does an unbound gas molecule near a water surface affect the water to which it will bind? What is the structure of hydrating waters in the surface region, and how does a hydrated solute molecule behave differently than as a gas? Experiments to address these questions provide valuable information, but have to date never fully characterize microscopic events and behaviors. However, these systems can be fully characterized computationally, and when coupled to the previous experimental work can provide a much more complete picture of gaseous adsorption to aqueous surfaces.

\suldiox~is a particularly important gas to use as a starting point and model system because of its importance in commercial and environmental systems. \cite{Boniface2000,Jayne1990,Johns2011,Heber1997,Faloona2009,Clegg2001} Its simple molecular structure, high solubility in water, and relative abundance make it a pivotal compound in numerous aqueous atmospheric reactions. A complete picture of the \suldiox/\wat~adsorption process will aide in understanding gaseous adsorption on the many aqueous surfaces in the environment, as well as in understanding the fundamental nature of gases in water's surface region. 

In this study we use ab initio quantum molecular dynamics (MD) techniques to model and simulate the hydrating structures that form around a surface-bound \suldiox~on water. We simulate a dynamic water surface, complete with all the extended hydrogen-bonding interactions that capture the variability of the \suldiox~hydrate structures, and the behavior of the water surface molecules. The quantum MD technique described herein allows more accurate and realistic simulation than our previous classical MD.\cite{Shamay2011} It is also superior to small cluster DFT studies because it does not assume geometry optimized configurations, and the extended interactions of a water slab are incorporated. In our previous classical MD study we determined net orientational behavior of \suldiox~binding to a water surface, and the orientation of the waters as they respond to the presence of an adsorbing gas. Understanding the orientational behavior of molecules in the aqueous interfacial region during adsorption was a necessary first step to understanding the specific details of gas-binding and surface behavior. 

%Unlike small cluster studies that use DFT calculations on geometry optimized structures to find likely geometries in a static vacuum setting, The quantum MD technique allows us to simulate the process much more realistically, and with greater accuracy than our previous classical MD work could allow.\cite{Shamay2011} This computational study greatly enhances the picture developed in our previous work on \suldiox. 

Quantum MD techniques are the logical follow-up as they accurately reproduce the hydration geometry around the bound \suldiox~molecules, and allow us to examine in detail the specific bonding interactions that occur within the surface hydrates, and in the extended bonding further into the water.\cite{Baer2010} Two parallel studies are performed in this work; one at 300K, and one at the more atmospherically relevant cold 273K. This set of temperatures complements our most recent experimental studies that showed the binding of gaseous \suldiox~to a water surface is greatly enhanced at cold temperatures.\cite{Ota2011} Other experiments by our group developed the picture of \suldiox~adsorption, and showed that \suldiox~surface hydrate complexes form when a water surface is exposed to \suldiox~gas.\cite{Tarbuck2005,Tarbuck2006} Although conclusions regarding the specific nature of those complexes could only be inferred from the experiments, our current computational studies now provide us with insights about the specific microscopic geometries and behaviors of the hydrating complexes.

We believe this to be the first temperature study using quantum MD to study the binding of small gas molecules on a water surface. We show how temperature affects the bonding behavior of the surface-adsorbed \suldiox~to neighboring waters, and propose a sequential binding mechanism for \suldiox~adsorbing to a water surface. We also examine \suldiox~binding behavior when bound to the surface waters. Lastly, our analysis of a specific bonding arrangement demonstrates an extended bonding structure of \suldiox~hydrates, as they are seen to preferentially form an extended cyclic ring structure through intermolecular bonds.
