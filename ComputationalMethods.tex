\section{Computational Methods}

On-the-fly ab initio molecular dynamics simulations were performed with the QUICKSTEP package, which is an implementation of the Gaussian plane wave method using the Kohn-Sham formulation of density functional theory (DFT).~\cite{VandeVondele2005} The Kohn-Sham orbitals are expanded using a linear combination of atom-centered Gaussian-type orbital functions. The electronic charge density was described using an auxiliary basis set of plane waves. Energies and forces from on-the-fly simulation sampling of the Born-Oppenheimer surface were calculated for each MD step using the Gaussian DZVP basis set, the exchange-correlation functional of Becke, Lee, Yang, and Parr (BLYP)~\cite{LEE1988}, and the atomic pseudo-potentials of the Goedecker, Teter, and Hutter type.~\cite{Goedecker1996} A simulation timestep of 1 fs was used, with a Nose-Hoover thermostat set at 273K and 300K for the ``cold'' and ``hot'' simulations, respectively. These computational parameters were verified to yield a reasonable description of bulk room temperature water when simulating a neat-water system. 

Initially, 10 equilibrated boxes of side-lengths 10.0\angs, with 36 randomly packed water molecules were used. Five of the boxes were used for each of the cold and hot simulations. A sulfur dioxide molecule was randomly placed onto the surface within 2.5\angs~of a water molecule centrally located above the waters in the z-axis. A copy of the initial system cubes were then expanded along one axis (z-axis) to 25\angs. The system energy was minimized through a geometry optimization. Subsequently, the system was equilibrated for 1 ns in canonical ensemble (NVT) conditions. Periodic boundaries were set on the two short axes to form an infinite slab. The equilibrated systems were then simulated for a further 20 ps in the microcanonical ensemble (NVE), with trajectory snapshots recorded every 1 fs. The initial 1 ns equilibration trajectory was not included in the final analysis. This resulted in 20,000 time steps of system trajectory for analysis in each of the hot and cold replicas of the system.
 
